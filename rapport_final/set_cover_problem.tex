%!TEX program = lualatex
\documentclass[12pt,letterpaper,twoside]{article}
\include{packages}
%!TEX root = ./set_cover_problem.tex

%=======================================================================================================
%=============================================== Informations ==========================================
%=======================================================================================================

% Cover infos
\title{Rapport final: problème de couverture d'ensemble}
\author{Benoît Cortier \& Maxime Pinard}
\date{\today{}}

% Fancy style options
\lhead{\small Benoît Cortier \& Maxime Pinard}
\rhead{\small Couverture d'ensemble}
\chead{}
\lfoot{}
\rfoot{}
\cfoot{\thepage}
\pagestyle{fancy}

%% Redefine the fancy plain page style
%\fancypagestyle{plain}{
%	\fancyhf{}
%	\lhead{\small Benoît Cortier \& Maxime Pinard}
%	\rhead{\small Couverture d'ensemble}
%	\chead{}
%	\lfoot{}
%	\rfoot{}
%	\cfoot{\thepage}
%}

%=======================================================================================================
%================================================== Configs ============================================
%=======================================================================================================

\usetikzlibrary{shapes}
\usetikzlibrary{arrows.meta}
\usetikzlibrary{calc}
\usetikzlibrary{positioning}
\usetikzlibrary{angles}
\usetikzlibrary{quotes}
\usetikzlibrary{decorations}

\definecolor{bg_color}{RGB}{250,250,229}
\definecolor{Cblue}{RGB}{38,75,150}
\definecolor{Cgreen}{RGB}{39,179,118}
\definecolor{Cdarkgreen}{RGB}{0,111,60}
\definecolor{Corange}{RGB}{249,167,62}
\definecolor{Cred}{RGB}{191,33,47}

\colorlet{good}{green!90!black}
\colorlet{average}{Corange}
\colorlet{bad}{Cred}


% Figures folder
\graphicspath{{figures/}}

% Prevent page breaks in paragraphs
\predisplaypenalty=1000
\postdisplaypenalty=1000
\clubpenalty=1000

% Minimal space required in the bottom margin not to move the title on the next page
\renewcommand{\bottomtitlespace}{.1\textheight}

% Links config, especialy for the table of contents
\hypersetup{
    colorlinks=true,
    linkcolor=black,
    urlcolor=blue,
    linktoc=all
}

% French language config
\frenchbsetup{StandardLayout=true,ReduceListSpacing=false,CompactItemize=false}

% Environments
\theoremstyle{definition}
\newtheorem{thm}{Théorème}
\newtheorem{defn}{Définition}
\newtheorem{prop}{Proposition}

% Tables
\newcolumntype{L}[1]{>{\raggedright\let\newline\\\arraybackslash\hspace{0pt}}m{#1}}
\newcolumntype{C}[1]{>{\centering\let\newline\\\arraybackslash\hspace{0pt}}m{#1}}
\newcolumntype{R}[1]{>{\raggedleft\let\newline\\\arraybackslash\hspace{0pt}}m{#1}}
\newcolumntype{Y}{>{\centering\arraybackslash}X}

% Listings
\lstset{
  %language=C,
  extendedchars=true,
  inputencoding=utf8,
  showlines=false,
  tabsize=4,
  showtabs=false,
  showspaces=false,
  showstringspaces=false,
  breaklines=true,
  breakatwhitespace=false,
  basicstyle=\footnotesize\ttfamily,
  frame=single,
  columns=[c]fixed,
  keepspaces=true,
}

%=======================================================================================================
%================================================= Functions ===========================================
%=======================================================================================================

% C++
\newcommand{\Cpp}{\texorpdfstring{C\kern-0.05em\protect\raisebox{.35ex}{\textsmaller[2]{+\kern-0.05em+}}}{C++}}

% Clear to the next left page
\newcommand*{\cleartoleftpage}{
  \clearpage \ifodd\value{page}\hbox{}\newpage\fi
}

% Paragraph with line break
\newcommand{\p}[1]{\paragraph{#1\\}}

% Function to print a warning sign
\newcommand{\dangersign}[1][2.5ex]
	{\renewcommand{\stacktype}{L}
		{\scaleto{\stackon[1pt]{\color{red}$\triangle$}{\fontsize{4pt}{4pt}\selectfont !}}{#1}}}

% Definition of some dt/dx/dy shortcuts for integrals
\newcommand{\dt}
{\;\mathrm{d}\,t}

\newcommand{\dx}
{\;\mathrm{d}\,x}

\newcommand{\dy}
{\;\mathrm{d}\,y}

% Definition of \Witem for 'itemize' environment with a warning sign
\newcommand{\Witem}
{\item[\dangersign{}]}

% Definition of a Max function shortcut
\newcommand{\Max}[2][ ]
{\underset{#1}{\text{Max}}\,#2}


\bibliography{references}
\nocite{*}

\begin{document}
	\maketitle{}
	\tableofcontents{}
	\newpage{}
	\section{Définition du problème}
		\subsection{SCP et WSCP}
			\paragraph*{Présentation\\}
				Le problème de couverture d'ensemble, ou \emph{Set Covering Problem} (SCP),
				fait parti des 21 problèmes NP-complets de \citeauthor{Karp1972}~\cite{Karp1972}
				et est NP-complet au sens fort \cite{garey2002computers}.
			\paragraph*{Problème de couverture d'ensemble\\}
				Étant donné un ensemble univers \(U = \{u_1, u_2, u_3, \dots, u_n\}\) et une famille \(S = \{s_1, s_2, \dots, s_m\}\) de sous-ensembles de \(U\),
				le problème consiste à trouver une sous-famille de \(S\) la plus petite possible permettant de couvrir chaque élément de \(U\)
				au moins une fois. Un élément \(e\) de \(U\) est couvert par un sous-ensemble \(A\) si \(e \in A\).
			\paragraph*{Problème de couverture d'ensemble pondéré\\}
				En associant un coût positif \(c_i\) à chaque sous-ensemble, on obtient le problème de couverture d'ensemble pondéré ou \emph{Weighted Set Covering Problem} (WSCP) et	l'objectif est alors de déterminer une couverture de coût minimum.~\cite{Vazirani2003}
		\subsection{Utilité}
			\paragraph*{}
				Une grande variété de problèmes de positionnement, de distribution, de planification et autres peuvent être formulés comme variantes du problème de couverture d'ensemble. Parmi les problèmes réels auxquels cette approche a été appliquée avec succès:\cite{Balas1982}
				\begin{itemize}
					\item problèmes de sélection de sites et d'allocation d'emplacement
					\item emplacement des installations des services d'urgence (casernes de pompiers, hôpitaux, etc.)
					\item choix de la taille et de l'emplacement des plates-formes de forage dans les champs pétrolifères en mer
					\item horaire des équipages pour les compagnies aériennes, les compagnies de bus, les chemins de fer
					\item répartition des fréquences de radiodiffusion entre stations de radio ou de télévision
					\item recherche d'informations (à partir de fichiers informatiques)
					\item \ldots
				\end{itemize}
	\section{Exemple minimal}
		\paragraph*{}
			Soit un ensemble univers \(U = \{u_1, u_2, \dots, u_{12}\}\) (représenté par des points sur le figure \ref{fig:example}) et une famille \(S = \{s_1, s_2, \dots, s_6\}\) de sous-ensembles de \(U\) (représentés par des rectangles sur la figure \ref{fig:example}) avec:
			\begin{itemize}
				\item \(s_1 = \{u_1, u_2, u_3, u_4, u_5, u_6\}\)
				\item \(s_2 = \{u_5, u_6, u_8, u_9\}\)
				\item \(s_3 = \{u_1, u_4, u_7, u_{10}\}\)
				\item \(s_4 = \{u_2, u_5, u_8, u_{11}\}\)
				\item \(s_5 = \{u_3, u_6, u_9, u_{12}\}\)
				\item \(s_6 = \{u_{10}, u_{11}, u_{12}\}\)
			\end{itemize}
		\paragraph*{}
			La solution optimale à cette instance est la sous famille \(S'=\{s_3, s_4, s_5\}\) (colorée en gris sur la figure \ref{fig:example}).
		\begin{figure}[H]
			\centering%
			\includegraphics[width=0.65\linewidth]{example}%
			\caption{Exemple d'instance du Set Cover Problem et solution optimale\cite{Mount2017}}%
			\label{fig:example}%
		\end{figure}
	\section{NP-complétude}
		\paragraph*{}
			FIXME: NP-difficile car problème d'optimisation et pas de décision. Vérifier NP-complet, class NP, NP-difficile, …
			Le Set Covering Problem a été démontré NP-complet par \citeauthor{Karp1972} en \citeyear{Karp1972} dans son article \citetitle{Karp1972}\cite{Karp1972}. Dans cet article, il montre que le Set Covering Problem peut être réduit au Vertex Cover Problem, qui peut être réduit au Clique Problem, qui peut lui-même être réduit au Boolean Satisfiability Problem.
			FIXME: justifier un peu plus. Déjà le Weighted c'est le Unweighted avec les poids à 1. Etc...
			https://en.wikipedia.org/wiki/NP-completeness
		\paragraph*{}
			Le théorème de Cook–Levin et sa démonstration publié en \citeyear{Cook1971} par \citeauthor{Cook1971} dans l'article \citetitle{Cook1971}\cite{Cook1971} prouve le Boolean Satisfiability Problem comme étant un problème NP-complet. Par réduction, le Set Covering Problem est donc aussi NP-complet.

	\section{Représentations}
		\subsection*{Représentation du problème}
			matrice \(a_{i,j} (n \times m)\)
		\subsection{Représentation des solutions}
			bitset

	\section{Instances du problème}
		\paragraph*{}
			On utilise les groupes d'instances mis a disposition par \citeauthor{OR-Library} dans son regroupement d'instances OR-Library\cite{OR-Library}. Parmis ces instances, celles de 4 à 6 proviennent l'article \citetitle{Balas1980}\cite{Balas1980} de \citeauthor{Balas1980}, celles de A à D proviennent de l'article \citetitle{Beasley1987}\cite{Beasley1987} de \citeauthor{Beasley1987} et celles de E à H proviennent de l'article \citetitle{Beasley1990}\cite{Beasley1990} de \citeauthor{Beasley1990}.
		\paragraph*{}
			Toutes les instances du problème de ces groupes on été générées en utilisant le shémas de \citeauthor{Balas1980}\cite{Balas1980} dans lequel le cout \(c_i\) de chaque colonne \(i\) est pris aléatoirement dans l'intervalle \(\llbracket0,100\rrbracket\), chaque colonne couvre au moins une ligne et chaque ligne est couverte par au moins deux clonnes.
		\paragraph*{}
		   Les propriétés de ces groupes d'instances sont décrites dans la table \ref{table:scp_problem_sets}, la densitée étant la proportion de \(1\) dans la matrice \(a_{i,j}\). La table \ref{table:problem_optimal_solutions}, contient les valeure optimales pour les problèmes pour lesquels elle est connue.
		\begin{table}[H]
			\centering
			\begin{tabular}{*{5}{C{65pt}}}
				\toprule
				Groupe d'instances & Nombre de lignes (\(m\)) & Nombre de colonnes (\(n\)) & densité (\%) & Nombre d'instances du groupe\\
				\midrule
				4 & 200 & 1000 & 2 & 10\\
				5 & 200 & 2000 & 2 & 10\\
				6 & 200 & 1000 & 5 & 5\\
				A & 300 & 3000 & 2 & 5\\
				B & 300 & 3000 & 5 & 5\\
				C & 400 & 4000 & 2 & 5\\
				D & 400 & 4000 & 5 & 5\\
				E & 500 & 5000 & 10 & 5\\
				F & 500 & 5000 & 20 & 5\\
				G & 1000 & 10000 & 2 & 5\\
				H & 1000 & 10000 & 5 & 5\\
				\bottomrule
			\end{tabular}
			\caption{Groupes d'instances du SCP utilisées\cite{OR-Library,Balas1980,Beasley1987,Beasley1990}}
			\label{table:scp_problem_sets}
		\end{table}
		\begin{table}[H]
			\centering
			\begin{minipage}[t]{0.45\linewidth}
				\centering
				\begin{tabular}{*{2}{C{65pt}}}
					\toprule
					Problem number & Optimal solution value\\
					\midrule
					4.1 & 429\\
					4.2 & 512\\
					4.3 & 516\\
					4.4 & 494\\
					4.5 & 512\\
					4.6 & 560\\
					4.7 & 430\\
					4.8 & 792\\
					4.9 & 641\\
					4.10 & 514\\
					\midrule
					5.1 & 253\\
					5.2 & 302\\
					5.3 & 226\\
					5.4 & 242\\
					5.5 & 211\\
					5.6 & 213\\
					5.7 & 293\\
					5.8 & 288\\
					5.9 & 279\\
					5.10 & 265\\
					\bottomrule
				\end{tabular}
			\end{minipage}
			\begin{minipage}[t]{0.45\linewidth}
				\centering
				\begin{tabular}{*{2}{C{65pt}}}
					\toprule
					Problem number & Optimal solution value\\
					\midrule
					A.1 & 253\\
					A.2 & 252\\
					A.3 & 232\\
					A.4 & 234\\
					A.5 & 236\\
					\midrule
					B.1 & 69\\
					B.2 & 76\\
					B.3 & 80\\
					B.4 & 79\\
					B.5 & 72\\
					\midrule
					C.1 & 227\\
					C.2 & 219\\
					C.3 & 243\\
					C.4 & 219\\
					C.5 & 215\\
					\midrule
					D.1 & 60\\
					D.2 & 66\\
					D.3 & 72\\
					D.4 & 62\\
					D.5 & 61\\
					\bottomrule
				\end{tabular}
			\end{minipage}
			\caption{Solutions optimales des instances du SCP utilisée\cite{Beasley1990}}
			\label{table:problem_optimal_solutions}
		\end{table}
	\section{Méthode exacte}
		\paragraph*{}
			Il existe une méthode exacte plus efficace qui utilise la méthode Branch and Bound ainsi que le Simplexe (pour résoudre une version
			du problème linéaire relaxée)~\cite{caprara2000algorithms}.
	\newpage\printbibliography[heading=bibintoc]{}
\end{document}

% Problem set        Files
% 4                  scp41, ..., scp410
% 5                  scp51, ..., scp510
% 6                  scp61, ..., scp65
% A                  scpa1, ..., scpa5
% B                  scpb1, ..., scpb5
% C                  scpc1, ..., scpc5
% D                  scpd1, ..., scpd5
% E                  scpe1, ..., scpe5

% Problem set        Files
% E                  scpnre1, ..., scpnre5
% F                  scpnrf1, ..., scpnrf5
% G                  scpnrg1, ..., scpnrg5
% H                  scpnrh1, ..., scpnrh5
